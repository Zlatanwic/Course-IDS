

\documentclass[12pt]{article}
\usepackage{graphicx} % Required for inserting images
\usepackage{amsmath}
\usepackage{minted}% 语法高亮和代码样式设置方面更加强大和灵活
\usepackage{epstopdf}
\usepackage{algorithm}
\usepackage{ctex}
\usepackage{xcolor}
\usepackage{listings}
\usepackage{tcolorbox}
\tcbuselibrary{skins}
\tcbuselibrary{minted}
\usemintedstyle{paraiso-dark}

\definecolor{codegreen}{rgb}{0,0.6,0}
\definecolor{codegray}{rgb}{0.5,0.5,0.5}
\definecolor{codepurple}{rgb}{0.58,0,0.82}
\definecolor{backcolour}{rgb}{0.95,0.95,0.92}

\lstdefinestyle{mystyle}{
    backgroundcolor=\color{backcolour},   
    commentstyle=\color{codegreen},
    keywordstyle=\color{magenta},
    numberstyle=\tiny\color{codegray},
    stringstyle=\color{codepurple},
    basicstyle=\ttfamily\footnotesize,
    breakatwhitespace=false,         
    breaklines=true,                 
    captionpos=b,                    
    keepspaces=true,                 
    numbers=left,                    
    numbersep=5pt,                  
    showspaces=false,                
    showstringspaces=false,
    showtabs=false,                  
    tabsize=2
}

\lstset{style=mystyle}
\title{初识MySQL}
\author{2353113 李阔}
\date{\today}
\usepackage{graphicx}
\usepackage{geometry}
\geometry{a4paper,left=2cm,right=2cm,top=2cm,bottom=3cm}


\begin{document}

\maketitle
\tableofcontents
\newpage
\section{MySQL的安装}
	{\songti 因为linux操作系统对于编程者的优越性,
		本报告的mysql安装部分主要介绍在linux虚拟机上安装MySQL,
		而Windows操作系统的MySQL的安装过程则会简要介绍。
		先介绍linux虚拟机的安装,我了解到个人用户安装linux操作系统一般会使用Ubuntu,于是我按照网上的各种教程配置了基于linux内核的Ubuntu虚拟机,设置如下图:

  %\includegraphics[height=15cm]{1.png}
  \begin{figure}[htbp]
\centering
\includegraphics[scale=0.4]{1.png}
\caption{虚拟机参数设置}
\label{figure}
\end{figure}
	}
 	{\songti 成功安装虚拟机后,在linux命令行窗口继续执行以下命令:
	}



\begin{lstlisting}[language=Bash]
        sudo apt-get updat
        sudo apt-get install mysql-server
        sudo apt-get install mysql-client

\end{lstlisting}


{\songti

%第一次在虚拟机登陆mysql时已设置了默认密码,需要手动查看   

    注:mysql-server 是MySQL核心程序将安装MySQL数据库服务器,用于生成管理多个数据库实例,持久保存数据并将其提供查询接口(SQL),供不同客户端调用。

mysql-client 是操作数据库实例的工具,允许连接到MySQL服务器使用该查询接口。它将为您提供MySQL命令行程序。

如果只需要连接到远程服务器并运行查询,只安装mysql-client就可以了。如果是服务器只提供连接服务的只需要安装mysql-server


}
\subsection{配置MySQL}
{\songti 运行初始化安全脚本}
\begin{lstlisting}[language=Bash]
        sudo mysql_secure_installation
\end{lstlisting}


\section{MySQL数据库的基本使用}
\subsection{启动MySQL数据库服务}

\begin{lstlisting}[language=Bash]
        sudo service mysql start
\end{lstlisting}
\subsection{停止MySQL数据库服务}
\begin{lstlisting}[language=Bash]
        sudo service mysql stop
\end{lstlisting}
\subsection{查看MySQL运行状态}
\begin{lstlisting}[language=Bash]
        sudo service mysql status
\end{lstlisting}

\section{MySQL创建用户与授权}
\subsection{使用root用户登陆}
{\songti 第一次以root用户在虚拟机登陆mysql时已设置了默认密码,需要手动查看,即在linux命令行窗口输入一下指令:}
\begin{lstlisting}[language=Bash]
        sudo cat /etc/mysql/debian.cnf
\end{lstlisting}
\begin{figure}[htbp]
    \centering
    \includegraphics[width=0.8\linewidth]{3.png}
    \caption{查找默认密码}
    \label{fig:enter-label}
\end{figure}

{\songti 记住密码后再以以下指令登陆:(输入在指令后会要求输入密码)}
\begin{lstlisting}[language=Bash]
        mysql -u root -p
\end{lstlisting}

\begin{figure}[htbp]
        \centering
        \includegraphics[width=0.8\linewidth]{2.png}
        \caption{登陆成功}
        \label{fig:enter-label}
\end{figure}
{\songti 可以将密码修改为个人常用密码}

\section{使用数据库}
\subsection{显示所有数据库}
         
\begin{tcblisting}{listing engine=minted,boxrule=0.1mm,
colback=blue!5!white,colframe=blue!75!black,
listing only,left=5mm,enhanced,sharp corners=all,
overlay={\begin{tcbclipinterior}\fill[red!20!blue!20!white] (frame.south west)
rectangle ([xshift=5mm]frame.north west);\end{tcbclipinterior}},
minted language=mysql,
minted style=tango,
minted options={fontsize=\small,breaklines,autogobble,linenos,numbersep=3mm}}
SHOW DATABASES;
\end{tcblisting}
\begin{figure}[htbp]
    \centering
    \includegraphics[width=0.5\linewidth]{Screenshot 2024-09-14 162915.png}
    \caption{Enter Caption}
    \label{fig:enter-label}
\end{figure}

\subsection{新建数据库}
\begin{tcblisting}{listing engine=minted,boxrule=0.1mm,
colback=blue!5!white,colframe=blue!75!black,
listing only,left=5mm,enhanced,sharp corners=all,
overlay={\begin{tcbclipinterior}\fill[red!20!blue!20!white] (frame.south west)
rectangle ([xshift=5mm]frame.north west);\end{tcbclipinterior}},
minted language=mysql,
minted style=tango,
minted options={fontsize=\small,breaklines,autogobble,linenos,numbersep=3mm}}
CREATE DATABASE TEST1;
\end{tcblisting}
\begin{figure}[htbp]
    \centering
    \includegraphics[width=0.75\linewidth]{Screenshot 2024-09-14 162936.png}
    \caption{数据库列表}
    \label{fig:enter-label}
\end{figure}
{\songti 再输入显示所有数据库命令,就可以看到我们新建的数据库出现在列表中:}
\begin{figure}[htbp]
    \centering
    \includegraphics[width=0.4\linewidth]{Screenshot 2024-09-14 163017.png}
    \caption{数据库列表}
    \label{fig:enter-label}
\end{figure}
\subsection{删除数据库}
\begin{tcblisting}{listing engine=minted,boxrule=0.1mm,
colback=blue!5!white,colframe=blue!75!black,
listing only,left=5mm,enhanced,sharp corners=all,
overlay={\begin{tcbclipinterior}\fill[red!20!blue!20!white] (frame.south west)
rectangle ([xshift=5mm]frame.north west);\end{tcbclipinterior}},
minted language=mysql,
minted style=tango,
minted options={fontsize=\small,breaklines,autogobble,linenos,numbersep=3mm}}
DROP DATABASE TEST1;
\end{tcblisting}
\begin{figure}[htbp]
    \centering
    \includegraphics[width=0.75\linewidth]{Screenshot 2024-09-15 122411.png}
    \caption{删除数据库}
    \label{fig:enter-label}
\end{figure}
{\songti 再输入SHOW DATABASDES;的语句就可以发现,我们刚刚创建的数据库已经被删除了}
\subsection{选择数据库}
\begin{tcblisting}{listing engine=minted,boxrule=0.1mm,
colback=blue!5!white,colframe=blue!75!black,
listing only,left=5mm,enhanced,sharp corners=all,
overlay={\begin{tcbclipinterior}\fill[red!20!blue!20!white] (frame.south west)
rectangle ([xshift=5mm]frame.north west);\end{tcbclipinterior}},
minted language=mysql,
minted style=tango,
minted options={fontsize=\small,breaklines,autogobble,linenos,numbersep=3mm}}
create DATABASE demo1;
use demo1;
\end{tcblisting}
\begin{figure}[htbp]
    \centering
    \includegraphics[width=0.5\linewidth]{Screenshot 2024-09-15 122708.png}
    \caption{创建并选择数据库}
    \label{fig:enter-label}
\end{figure}

\subsection{创建数据表}
\begin{tcblisting}{listing engine=minted,boxrule=0.1mm,
colback=blue!5!white,colframe=blue!75!black,
listing only,left=5mm,enhanced,sharp corners=all,
overlay={\begin{tcbclipinterior}\fill[red!20!blue!20!white] (frame.south west)
rectangle ([xshift=5mm]frame.north west);\end{tcbclipinterior}},
minted language=mysql,
minted style=tango,
minted options={fontsize=\small,breaklines,autogobble,linenos,numbersep=3mm}}
CREATE TABLE users (
    id INT AUTO_INCREMENT PRIMARY KEY,
    username VARCHAR(50) NOT NULL,
    birthdate DATE,
    is_active BOOLEAN DEFAULT TRUE
);
\end{tcblisting}
\begin{figure}[htbp]
    \centering
    \includegraphics[width=0.8\linewidth]{Screenshot 2024-09-15 124246.png}
    \caption{创建数据表}
    \label{fig:enter-label}
\end{figure}
{\songti 
id: 用户 id,整数类型,自增长,作为主键 \\
username: 用户名,变长字符串,不允许为空 \\
birthdate: 用户的生日,日期类型 \\
is\_active: 用户是否已经激活,布尔类型,默认值为 true \\
}
\subsection{删除数据表}
\begin{tcblisting}{listing engine=minted,boxrule=0.1mm,
colback=blue!5!white,colframe=blue!75!black,
listing only,left=5mm,enhanced,sharp corners=all,
overlay={\begin{tcbclipinterior}\fill[red!20!blue!20!white] (frame.south west)
rectangle ([xshift=5mm]frame.north west);\end{tcbclipinterior}},
minted language=mysql,
minted style=tango,
minted options={fontsize=\small,breaklines,autogobble,linenos,numbersep=3mm}}
drop TABLE users;
\end{tcblisting}
\subsection{插入数据}
\subsubsection{插入单个数据}
\begin{tcblisting}{listing engine=minted,boxrule=0.1mm,
colback=blue!5!white,colframe=blue!75!black,
listing only,left=5mm,enhanced,sharp corners=all,
overlay={\begin{tcbclipinterior}\fill[red!20!blue!20!white] (frame.south west)
rectangle ([xshift=5mm]frame.north west);\end{tcbclipinterior}},
minted language=mysql,
minted style=tango,
minted options={fontsize=\small,breaklines,autogobble,linenos,numbersep=3mm}}
INSERT INTO users (username,  birthdate, is_active)
VALUES ('test1', '1990-01-01', true);
\end{tcblisting}
\begin{figure}[htbp]
    \centering
    \includegraphics[width=0.75\linewidth]{Screenshot 2024-09-15 130855.png}
    \caption{插入单个数据}
    \label{fig:enter-label}
\end{figure}
\subsubsection{插入多组数据}
\begin{tcblisting}{listing engine=minted,boxrule=0.1mm,
colback=blue!5!white,colframe=blue!75!black,
listing only,left=5mm,enhanced,sharp corners=all,
overlay={\begin{tcbclipinterior}\fill[red!20!blue!20!white] (frame.south west)
rectangle ([xshift=5mm]frame.north west);\end{tcbclipinterior}},
minted language=mysql,
minted style=tango,
minted options={fontsize=\small,breaklines,autogobble,linenos,numbersep=3mm}}
INSERT INTO users (username,  birthdate, is_active)
VALUES
    ('test2',  '1985-07-10', true),
    ('test3',  '1988-11-25', false),
    ('test4',  '1993-05-03', true);
\end{tcblisting}
\begin{figure}[htbp]
    \centering
    \includegraphics[width=0.75\linewidth]{Screenshot 2024-09-15 131109.png}
    \caption{插入多组数据}
    \label{fig:enter-label}
\end{figure}
\subsection{查找数据}
\begin{tcblisting}{listing engine=minted,boxrule=0.1mm,
colback=blue!5!white,colframe=blue!75!black,
listing only,left=5mm,enhanced,sharp corners=all,
overlay={\begin{tcbclipinterior}\fill[red!20!blue!20!white] (frame.south west)
rectangle ([xshift=5mm]frame.north west);\end{tcbclipinterior}},
minted language=mysql,
minted style=tango,
minted options={fontsize=\small,breaklines,autogobble,linenos,numbersep=3mm}}
-- 使用where子句
SELECT * FROM users WHERE is_active = true;
\end{tcblisting}
\begin{figure}[htbp]
    \centering
    \includegraphics[width=0.75\linewidth]{Screenshot 2024-09-15 132006.png}
    \caption{查找激活状态的用户}
    \label{fig:enter-label}
\end{figure}

\begin{tcblisting}{listing engine=minted,boxrule=0.1mm,
colback=blue!5!white,colframe=blue!75!black,
listing only,left=5mm,enhanced,sharp corners=all,
overlay={\begin{tcbclipinterior}\fill[red!20!blue!20!white] (frame.south west)
rectangle ([xshift=5mm]frame.north west);\end{tcbclipinterior}},
minted language=mysql,
minted style=tango,
minted options={fontsize=\small,breaklines,autogobble,linenos,numbersep=3mm}}
-- 使用or逻辑符
SELECT * FROM users WHERE is_active = TRUE OR birthdate < '1990-01-01';
\end{tcblisting}
\begin{figure}[htbp]
    \centering
    \includegraphics[width=0.75\linewidth]{Screenshot 2024-09-15 132807.png}
    \caption{使用or查询}
    \label{fig:enter-label}
\end{figure}

\subsection{UPDATE更新}
\begin{tcblisting}{listing engine=minted,boxrule=0.1mm,
colback=blue!5!white,colframe=blue!75!black,
listing only,left=5mm,enhanced,sharp corners=all,
overlay={\begin{tcbclipinterior}\fill[red!20!blue!20!white] (frame.south west)
rectangle ([xshift=5mm]frame.north west);\end{tcbclipinterior}},
minted language=mysql,
minted style=tango,
minted options={fontsize=\small,breaklines,autogobble,linenos,numbersep=3mm}}
UPDATE users
SET is_active=false
WHERE username='test1';
\end{tcblisting}

\begin{figure}[htbp]
    \centering
    \includegraphics[width=0.75\linewidth]{Screenshot 2024-09-15 133737.png}
    \caption{将test1用户的激活状态更新为false}
    \label{fig:enter-label}
\end{figure}

{\songti 然后我们再查询激活状态为false的用户,则包括了test1,说明我们的update生效了}

\begin{figure}[htbp]
    \centering
    \includegraphics[width=0.75\linewidth]{Screenshot 2024-09-15 133745.png}
    \caption{显示激活状态为false的用户}
    \label{fig:enter-label}
\end{figure}
\newpage
\section{总结}
{\songti 
    坦白说,这次报告的前期还是比较波折的,涉及到操作系统的选择,以及排版工具的选择。考虑到之前没有怎么接触过linux也没有装过虚拟机,便想以此次作业为契机,在我的电脑上安装linux内核的虚拟机,而操作系统的选择,我曾问过学长,自己也在网上查询过资料,最终在centOS和Ubuntu之间选择了Ubuntu,因为我了解到centOS较多用于企业服务器端,而初linux初学者以及个人用户一般更倾向于使用Ubuntu,因为Ubuntu有比较简单的安装过程和图形化界面以及社区论坛,对新手更加友好。\\
    
    而排版工具的选择也是摇摆不定,一开始选择latex,因为其美观并且我之前曾使用过这两大原因我开始着手在本地的latex编译器Texstudio上进行latex的编译工作,而由于很长时间不使用latex,发现我在有一些排版想法时却不能实现,出现各种报错,于是我开始逐渐放弃使用latex排版,进而选择学习成本更低更加轻量的markdown语言进行排版,而上手后却发现版面实在是不太美观,而我意识到语气花费一定的学习成本去使mardown排版达到一定的美观程度,不如使用latex,如果能比较熟练地掌握latex,其在以后的其他课的报告撰写上将受益无穷,所以我又转而使用了在线的latex编译器overleaf内嵌了XeLatex。\\
    
    而基本的MySQL语句的学习到没有遇到太大的障碍,就是第一次登陆时查找初始默认的密码时看了不少教程,花费了我不少时间。\\
    
    而目前,以我的个人理解,我了解到类似于MySQL这样数据库的应用要与前端的接口进行匹配,比如作为一个网站或者一个小程序的后端,而这些都是我的盲区,所以我现在的课余学习目标就暂时为学习一定简单的前端技术,并着重学习前后端的接口以及后端的数据处理。\\

    这次报告只是载体,而他真正地让我在撰写报告中学会很多东西,比如linux虚拟机的基本知识,MySQl的基本语句等等。
    
    


}



\end{document}